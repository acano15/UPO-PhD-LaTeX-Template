% called by main.tex
%
\clearpage
\pagenumbering{roman}
\setcounter{page}{1}

\chapter*{Abstract} \label{chap::abstract}

<Abstract in English: maximum of 4000 characters, plain text (without symbols), structured summary of the thesis (introduction or motivation, objectives, findings and conclusions)>

Abstracts provide a concise summary of an academic work, helping readers quickly determine if the paper is relevant to their research. According to the San José State University Writing Center \cite{SJSUWritingAbstract}, abstracts generally consist of five parts: introduction, purpose, method, result, and conclusion. These parts each serve a distinct purpose, from introducing the research context to presenting key findings and their implications. Abstracts should be brief, typically 100 to 300 words, and are often required for conference submissions and funding applications.

An effective abstract clearly outlines the research question, describes the methodology used, summarizes the main findings, and suggests the broader significance of the results. While abstracts may vary slightly by discipline, their primary function is to allow readers to understand the scope and value of the research without reading the full paper \cite{SJSUWritingAbstract}.

\chapter*{Resumen} \label{chap::resumen}

<Resumen en español: máximo de 4000 caracteres, texto plano (sin símbolos), resumen estructurado de la tesis (introducción o motivación, objetivos, hallazgos y conclusiones)>

Los resúmenes proporcionan un resumen conciso de un trabajo académico, ayudando a los lectores a determinar rápidamente si el artículo es relevante para su investigación. Según el Writing Center de la Universidad Estatal de San José \cite{SJSUWritingAbstract}, los resúmenes generalmente consisten en cinco partes: introducción, propósito, método, resultado y conclusión. Cada una de estas partes tiene una función distinta, desde introducir el contexto de la investigación hasta presentar los hallazgos clave y sus implicaciones. Los resúmenes deben ser breves, generalmente de 100 a 300 palabras, y a menudo se requieren para presentaciones en conferencias y solicitudes de financiamiento.

Un resumen efectivo describe claramente la pregunta de investigación, detalla la metodología utilizada, resume los hallazgos principales y sugiere la relevancia más amplia de los resultados. Aunque los resúmenes pueden variar ligeramente según la disciplina, su función principal es permitir que los lectores comprendan el alcance y valor de la investigación sin necesidad de leer el documento completo \cite{SJSUWritingAbstract}.
