\setcounter{chapter}{4}
\chapter{Discussion and Future Lines} \label{chap::disscusionFutureLines}
\section{How to write the discussion}
The discussion section is an essential part of the research paper, where the author analyzes, interprets, and explains the findings in relation to the research question(s). According to the San José State University Writing Center \cite{SJSUWritingCenterDiscussion}, the purpose of this section is to place the results in context, explain their significance, and address any unexpected findings. Furthermore, it is important to acknowledge the study's limitations and suggest areas for future research.

An effective discussion begins by summarizing the key findings and linking them to the research question. The findings are then compared with the existing literature to demonstrate how they align or diverge from previous studies. Any unexpected results should be discussed and explained, as they may provide new insights into the research problem. Limitations of the study, such as data constraints or methodological weaknesses, should be transparently addressed to enhance the credibility of the research \cite{SJSUWritingCenterDiscussion}.

In terms of future research, it is essential to propose follow-up studies that can further explore gaps identified in the current research. These recommendations should be concise and focused on the most pressing questions that remain unanswered. Concluding the discussion, the key findings and their broader implications are restated, underscoring the importance of the research in contributing to the field and suggesting potential practical applications \cite{SJSUWritingCenterDiscussion}.
