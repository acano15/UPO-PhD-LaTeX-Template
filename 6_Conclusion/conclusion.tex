\setcounter{chapter}{5}
\chapter{Conclusion} \label{chap::conclusion}
\section{How to write the conclusion}
The conclusion is a critical component of any research paper, as it is the final opportunity to leave a lasting impression on the reader. According to the San José State University Writing Center \cite{SJSUWritingCenterConclusion}, a well-written conclusion should restate and synthesize the main points of the paper, but without simply repeating what has already been said. Instead, the focus should be on emphasizing how the research has contributed to the reader’s understanding of the topic.

A strong conclusion adds significance to the discussion by connecting the findings to the real world, offering insights into the broader implications of the research. Furthermore, it is important to answer the "So what?" question, explaining why the research matters and how it affects the field or related areas. This ensures that the reader understands the relevance and practical applications of the work \cite{SJSUWritingCenterConclusion}.

Finally, the conclusion may include a call to action or a powerful closing statement that leaves a lasting impact. This can be achieved by summarizing the key contributions of the paper and proposing future steps, ensuring that the research remains memorable and meaningful for the audience \cite{SJSUWritingCenterConclusion}.
