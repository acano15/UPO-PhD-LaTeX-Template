

\setcounter{chapter}{3}
\chapter{Experiments and Results} \label{chap::experimentsResult}
\section{How to write the experiment and result section}
The results section is a key component of any research paper, as it presents the findings of the study in response to the research question(s). According to the San José State University Writing Center \cite{SJSUWritingCenterResults}, the purpose of this section is to provide the data in a structured and unbiased manner, without attempting to analyze or interpret it. The results can be presented in text, tables, or figures, but the data should always be connected to the research question to maintain relevance.

In presenting the results, it is crucial to follow a logical structure, often organizing the findings either thematically or chronologically. Each set of findings should be clearly introduced to keep the reader’s focus aligned with the study's objectives. Visual aids such as tables and graphs can be used to enhance the presentation but should not replace the textual description of the results \cite{SJSUWritingCenterResults}. 

An effective results section concludes by summarizing the key findings, preparing the reader for the subsequent discussion section, where the implications of the results are analyzed. By adhering to this structure, the results are conveyed clearly and effectively, providing a solid foundation for further interpretation \cite{SJSUWritingCenterResults}.
