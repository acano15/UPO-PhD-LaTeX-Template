\setcounter{chapter}{1}
\chapter{Related work} \label{relatedWork}

\section{How to write the related work section}
In preparing a related work section, a strategic approach can greatly reduce the time spent searching and reading while maximizing the depth and relevance of the literature. According to Animatas \cite{Animatas2023}, the key steps involve identifying seed papers, expanding and filtering the pool, and finally organizing the related work in a meaningful way.

The process begins by identifying a few initial seed papers, typically from Google Scholar, which serve as a foundation. These papers are added to a spreadsheet, where essential details such as title, authors, and inclusion status are tracked. The next step involves expanding the pool by reviewing citations in the seed papers and adding any relevant ones from the last three years. This expanded list is then filtered by reading abstracts and titles, while noting down reasons for exclusion \cite{Animatas2023}.

Once a core set of relevant papers is identified, the author extracts important keywords from the papers to construct search queries for platforms like Web of Science and Scopus. This helps ensure comprehensive coverage of the literature. The filtered results are carefully documented in the spreadsheet, and the process continues until a solid overview of the literature is achieved. Citation alerts are set up to remain up-to-date with new developments throughout the project \cite{Animatas2023}.

The final step involves organizing the related work section by clustering papers into similar topics, creating brief summaries for each, and weaving them into cohesive paragraphs. This structured approach ensures that the related work section is well-informed and demonstrates how the current research builds on or addresses gaps in previous work \cite{Animatas2023}.
