\chapter*{Anexo}
\section*{Example pyhthon code 1}\label{anexo1}
\begin{lstlisting}[language=Python]

import os

class ThesisTemplate:
	def __init__(self, author_name):
		self.author_name = author_name

	def print_greeting(self):
		message = f"Hello, World! I am a template for the thesis authored by {self.author_name}."
		border = "*" * len(message)
		print(border)
		print(message)
		print(border)

if __name__ == "__main__":
	author = "Your Name"  # Replace with your actual name
	template = ThesisTemplate(author)
	template.print_greeting()
\end{lstlisting}


\section*{Example pyhthon code 2}\label{anexo2}
\begin{lstlisting}[language=Python]
import random
import numpy as np

class SampleTemplate:
	def __init__(self):
		self.operation = self.generate_random_operation()

	def generate_random_operation(self):
		# Generate two random integers
		num1 = random.randint(1, 10)
		num2 = random.randint(1, 10)
		
		# Choose a random operation 
		operations = ['+', '-', '*', '/']
		operation = random.choice(operations)
		
		# Calculate result based on operation
		if operation == '+':
		result = num1 + num2
		elif operation == '-':
		result = num1 - num2
		elif operation == '*':
		result = num1 * num2
		elif operation == '/':
		# Prevent division by zero
		if num2 == 0:
		num2 = 1  # Change num2 to 1 to avoid division by zero
		result = num1 / num2
		
		return f"{num1} {operation} {num2} = {result}"

	def greet(self):
		print("Hello World! I am a template for the thesis.")
		print("Here's a random numeric operation for you:")
		print(self.operation)

# Example of how to use the class
if __name__ == "__main__":
	template = SampleTemplate()
	template.greet()
	


\end{lstlisting}