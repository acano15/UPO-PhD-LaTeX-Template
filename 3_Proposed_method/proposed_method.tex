
\setcounter{chapter}{2}
\chapter{Proposed Method} \label{chap::proposedMethod}
\section{How to write the proposed method section}
The methodology section outlines how the research was conducted and provides the necessary details for readers to assess the validity and reliability of the study. According to the San José State University Writing Center \cite{SJSUWritingCenterProposedMethod}, this section should describe the research question, the type of data used, and why this data is appropriate for the research. It also involves explaining the data collection process, including the tools, materials, sampling criteria, and the size of the sample.

In quantitative research, the methodology section includes the types of mathematical analyses conducted, whereas qualitative research focuses on identifying patterns in language, theme, or structure from non-numerical data. Additionally, it is important to justify the selected methodology by explaining why it is suitable for the research question and by addressing any challenges encountered \cite{SJSUWritingCenterProposedMethod}.

For the proposed research, data collection involved [insert your data collection methods, e.g., gathering audio-visual data], followed by the implementation of [insert specific tools or technologies used, e.g., machine learning models, audio processing tools]. The criteria for selecting data sources focused on [insert criteria]. The data analysis process involved [insert analysis methods, such as statistical techniques or qualitative coding], ensuring that the analysis aligned with the research objectives. Tools such as [insert tools, e.g., Python, TensorFlow] were used for data processing and interpretation.
